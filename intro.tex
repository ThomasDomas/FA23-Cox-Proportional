% Options for packages loaded elsewhere
\PassOptionsToPackage{unicode}{hyperref}
\PassOptionsToPackage{hyphens}{url}
\PassOptionsToPackage{dvipsnames,svgnames,x11names}{xcolor}
%
\documentclass[
  letterpaper,
  DIV=11,
  numbers=noendperiod]{scrreprt}

\usepackage{amsmath,amssymb}
\usepackage{iftex}
\ifPDFTeX
  \usepackage[T1]{fontenc}
  \usepackage[utf8]{inputenc}
  \usepackage{textcomp} % provide euro and other symbols
\else % if luatex or xetex
  \usepackage{unicode-math}
  \defaultfontfeatures{Scale=MatchLowercase}
  \defaultfontfeatures[\rmfamily]{Ligatures=TeX,Scale=1}
\fi
\usepackage{lmodern}
\ifPDFTeX\else  
    % xetex/luatex font selection
\fi
% Use upquote if available, for straight quotes in verbatim environments
\IfFileExists{upquote.sty}{\usepackage{upquote}}{}
\IfFileExists{microtype.sty}{% use microtype if available
  \usepackage[]{microtype}
  \UseMicrotypeSet[protrusion]{basicmath} % disable protrusion for tt fonts
}{}
\makeatletter
\@ifundefined{KOMAClassName}{% if non-KOMA class
  \IfFileExists{parskip.sty}{%
    \usepackage{parskip}
  }{% else
    \setlength{\parindent}{0pt}
    \setlength{\parskip}{6pt plus 2pt minus 1pt}}
}{% if KOMA class
  \KOMAoptions{parskip=half}}
\makeatother
\usepackage{xcolor}
\setlength{\emergencystretch}{3em} % prevent overfull lines
\setcounter{secnumdepth}{-\maxdimen} % remove section numbering
% Make \paragraph and \subparagraph free-standing
\ifx\paragraph\undefined\else
  \let\oldparagraph\paragraph
  \renewcommand{\paragraph}[1]{\oldparagraph{#1}\mbox{}}
\fi
\ifx\subparagraph\undefined\else
  \let\oldsubparagraph\subparagraph
  \renewcommand{\subparagraph}[1]{\oldsubparagraph{#1}\mbox{}}
\fi


\providecommand{\tightlist}{%
  \setlength{\itemsep}{0pt}\setlength{\parskip}{0pt}}\usepackage{longtable,booktabs,array}
\usepackage{calc} % for calculating minipage widths
% Correct order of tables after \paragraph or \subparagraph
\usepackage{etoolbox}
\makeatletter
\patchcmd\longtable{\par}{\if@noskipsec\mbox{}\fi\par}{}{}
\makeatother
% Allow footnotes in longtable head/foot
\IfFileExists{footnotehyper.sty}{\usepackage{footnotehyper}}{\usepackage{footnote}}
\makesavenoteenv{longtable}
\usepackage{graphicx}
\makeatletter
\def\maxwidth{\ifdim\Gin@nat@width>\linewidth\linewidth\else\Gin@nat@width\fi}
\def\maxheight{\ifdim\Gin@nat@height>\textheight\textheight\else\Gin@nat@height\fi}
\makeatother
% Scale images if necessary, so that they will not overflow the page
% margins by default, and it is still possible to overwrite the defaults
% using explicit options in \includegraphics[width, height, ...]{}
\setkeys{Gin}{width=\maxwidth,height=\maxheight,keepaspectratio}
% Set default figure placement to htbp
\makeatletter
\def\fps@figure{htbp}
\makeatother

\KOMAoption{captions}{tableheading}
\makeatletter
\makeatother
\makeatletter
\makeatother
\makeatletter
\@ifpackageloaded{caption}{}{\usepackage{caption}}
\AtBeginDocument{%
\ifdefined\contentsname
  \renewcommand*\contentsname{Table of contents}
\else
  \newcommand\contentsname{Table of contents}
\fi
\ifdefined\listfigurename
  \renewcommand*\listfigurename{List of Figures}
\else
  \newcommand\listfigurename{List of Figures}
\fi
\ifdefined\listtablename
  \renewcommand*\listtablename{List of Tables}
\else
  \newcommand\listtablename{List of Tables}
\fi
\ifdefined\figurename
  \renewcommand*\figurename{Figure}
\else
  \newcommand\figurename{Figure}
\fi
\ifdefined\tablename
  \renewcommand*\tablename{Table}
\else
  \newcommand\tablename{Table}
\fi
}
\@ifpackageloaded{float}{}{\usepackage{float}}
\floatstyle{ruled}
\@ifundefined{c@chapter}{\newfloat{codelisting}{h}{lop}}{\newfloat{codelisting}{h}{lop}[chapter]}
\floatname{codelisting}{Listing}
\newcommand*\listoflistings{\listof{codelisting}{List of Listings}}
\makeatother
\makeatletter
\@ifpackageloaded{caption}{}{\usepackage{caption}}
\@ifpackageloaded{subcaption}{}{\usepackage{subcaption}}
\makeatother
\makeatletter
\@ifpackageloaded{tcolorbox}{}{\usepackage[skins,breakable]{tcolorbox}}
\makeatother
\makeatletter
\@ifundefined{shadecolor}{\definecolor{shadecolor}{rgb}{.97, .97, .97}}
\makeatother
\makeatletter
\makeatother
\makeatletter
\makeatother
\ifLuaTeX
  \usepackage{selnolig}  % disable illegal ligatures
\fi
\IfFileExists{bookmark.sty}{\usepackage{bookmark}}{\usepackage{hyperref}}
\IfFileExists{xurl.sty}{\usepackage{xurl}}{} % add URL line breaks if available
\urlstyle{same} % disable monospaced font for URLs
\hypersetup{
  colorlinks=true,
  linkcolor={blue},
  filecolor={Maroon},
  citecolor={Blue},
  urlcolor={Blue},
  pdfcreator={LaTeX via pandoc}}

\author{}
\date{}

\begin{document}
\ifdefined\Shaded\renewenvironment{Shaded}{\begin{tcolorbox}[boxrule=0pt, enhanced, interior hidden, borderline west={3pt}{0pt}{shadecolor}, breakable, sharp corners, frame hidden]}{\end{tcolorbox}}\fi

\hypertarget{introduction}{%
\chapter{Introduction}\label{introduction}}

The Application of the Cox Proportional Hazard Model and Its Application
and Effectiveness

The Cox Proportional Hazard Model, also known as the Cox Regression or
Cox PH model is an important statistical tool used in survival analysis.
This model was developed in 1972 by a British statistician by the name
of Sir David R. Cox to analyze time-to-event data. Survival analysis,
such as the Cox Regression is a time-to-event analysis, used to
investigate the time until the occurrence of an event of interest.
Applications of this regression model go far beyond statistical setting
and are widely used in medical research and are growing in use in
reliability engineering and other outcomes research. In this
introduction, we will review Cox Regression, delve deeper into how the
Cox Regression has been applied to three different medical research
papers, note some of its uses outside of the medical field, and discuss
how effective this statistical analysis is.

Prior to the publication of the Cox Proportional Hazard model, tests
such as the log-rank test were used but limited to observing the effect
of one variable at a time {[}1{]}. Additionally, the logrank test could
not quantify the effect of the variables {[}1{]}. The Cox proportional
hazards model estimates a hazard as {[}1{]}:

\(h(t)=h_0(t) \exp \left(B_1 X_1+B_2 X_2+\ldots B_p X_p\right)\)

\(h\) is the hazard at time \(t\), \(h_0(t)\) is the baseline hazard,
\(\mathrm{X}\) are the covariates.

The Cox proportional hazard model assumes the hazard rate is a product
of a baseline hazard rate and that that baseline hazard rate is the
hazard rate when the covariates have no effect on the event. A
regression vector is estimated by maximizing a marginal, partial, or
maximum likelihood function. The significance of the covariates can be
tested with analytical methods (log rank test, or chi-squared test) or
graphical methods (cumulative hazard rate plots or residual plots).
Insignificant covariates can be removed and a regression vector
recalculated {[}2{]}.

Dividing both sides by \(h_0(t)\) yields the hazard ratio

\(\mathrm{h}(\mathrm{t}) / \mathrm{h}_{\mathrm{o}}(\mathrm{t})=\exp \left(\mathrm{B}_1 \mathrm{X}_1+\mathrm{B}_2 \mathrm{X}_2+\ldots \mathrm{B}_{\mathrm{p}} \mathrm{X}_{\mathrm{p}}\right) \text {. }\)

To meet the model assumptions this hazard ratio must remain constant
over time. Babinska, et al.~emphasize this in their paper looking at the
Cox PH model's use in analyzing the outcomes of patients with acute
coronary syndrome {[}3{]}. Cox proportional hazard model also assumes
events are independent and that censoring is uninformative {[}1{]}. That
is that the time to event for one person (or thing) is independent of
the time to event for another person (or thing) and people (or things)
who do not continue to the end of the study, aka. are censored, have the
same risk of the event as those that do continue until the end of the
study period.

Reasons the Cox proportional hazards model is popular is that it does
not require the baseline hazard, \(h_0(t)\) to be known {[}1{]}.

Babinska, et al.~look into the limitations of Cox Regression in
predicting mortality rates in patients with acute coronary syndrome
(ACS) {[}3{]}. Although there have been significant advances in medicine
and interventional therapies, cardiovascular disease (CVD) still burdens
the medical system. The researchers see the importance of Cox Regression
but worry about a potential violation of the proportionality of hazard
assumption using a simple Cox regression model {[}3{]}. The study at
first examined if proportional hazard assumptions are met in various
factors including homocysteine and sodium concentrations. These
assumptions are important for accurate Cox regression analysis because
they ensure that hazard ratios remain constant. The study found there
was a violation in some of the cases, such as homocysteine and sodium
concentrations. Without verification of these assumptions, the increased
risk of death due to ACS might be incorrectly identified. The study
looked into one hundred and fifty consecutive patients with acute
coronary syndrome (ACS) with a mean age of 65 (range; 21-92 years), 33
(22\%) deaths was registered during the follow-up period of 64 months
{[}3{]}. In the Cox regression model, several factors were identified as
risk factors for long-term prognosis in patients with ACS. The factors
included smoking, the presence of diabetes and anemia, the duration of
coronary artery disease, and abnormal serum concentrations of uric acid,
sodium, homocysteine, cystatine C, and NT-proBNP. The study stressed the
importance of ensuring that the proportional hazard assumption is met in
the Cox regression. Failure to meet the assumption can lead to false
positives in the conclusion and incorrect identification of factors
leading to ACS-related deaths. To remedy the violation, the study
recommends using the Cox stratified regression model or extended
regression models with time-dependent variables.

The next study investigated the impact of regional medical disparities
on complications in patients with hypertension while using the Cox
Proportional Hazard model. The study took place in South Korea since it
has become the fastest-aging country worldwide, and there is a greater
likelihood of an increase in the prevalence of hypertension {[}4{]}. As
of 2022, Seoul, the capital of South Korea, had 4.8 doctors per 1,000
inhabitants, and except for metropolitan cities (Busan, Daegu), all the
other areas had an average of less than 3.2 doctors per 1,000
inhabitants {[}5{]}. The disparities between the number of physicians in
the more populated regions and the rural areas play an important role in
hypertension compilations. The study's results aligned with similar
studies conducted in China, Romania, and the United States. These
studies also reported a correlation between disparities in hypertension
treatment and control, particularly in rural areas. The study also
suggested that the higher risk of hypertension complications in
medically vulnerable regions may be due to disparities in access to
healthcare. The study included a cohort of 246,490 participants, with
the goal of investigating the interaction between residential regions
(vulnerable vs.~non-vulnerable) and diagnosis areas (outside vs.~inside)
in relation to complications in patients. The participants were divided
into four groups --- vulnerable and outside region, vulnerable and
inside region, non-vulnerable and outside region, and non- vulnerable
and inside region. Multivariate Cox regression analysis was fitted. The
interaction term between the residential region (vulnerable
vs.~non-vulnerable) and the diagnosis area (outside vs.~inside) was
significant in the adjusted model (p-value =0.005) {[}4{]}. The Cox
regression showed that individuals living in vulnerable and outside
regions had the highest risk of complications. This analysis was
supported by the hazard ratio (HR) in which vulnerable and outside
regions had the highest ratio of 1.l56. The vulnerable and outside
region group had the highest rate of complications for cardiovascular
and cerebrovascular diseases when compared to the reference group.
However, for kidney disease, the non-vulnerable and outside region group
had the highest rate of complications.

In this third article, researchers are using Cox regression in addition
to Artificial Neural Network (ANN) to postoperative mortality after hip
fracture surgery. Postoperative mortality is any death, regardless of
cause thirty-days after surgery. As the population is aging more hip
fracture has become a public health issue {[}6{]}. The substantial
associated medical costs in the United Kingdom (UK) and Taiwan have
increased for these procedures. This is why understanding and predicting
the underlying causes of postoperative mortality after hip surgeries is
of great importance. Traditional parameter models have not been reliable
enough, ANN and Cox regression models are used due to the fact they are
the most common models used in the healthcare industry for predicting
postoperative mortality. The ANN model captures nonlinear interactions
among risk factors, making it favorable in medical decision-making and
mortality predictions. Cox proportional hazard (PH) model is standard in
survival analysis but has a limitation in predicting longitudinal
survival. The purpose of the study is to use ANN and Cox models to
identify influential predictors of postoperative mortality after hip
fracture and conduct a global sensitivity analysis to assess important
predictors. In total, researchers analyzed 10,534 hip fracture
procedures. During this period, 71.2\% hip fracture patients were
referred to lower- level medical institutions for rehabilitation after
surgery, and 28.8\% patients continued treatment at the same medical
institution {[}7{]}. The mean age of patients was 68.3 (SD 14.6) years,
with females representing 57.6\% of the patients {[}7{]}. The training
data set was 7374 cases, the testing data set of 1580 cases and the same
number of 1580 for the validation dataset. The study found that the
artificial neural networks (ANN) model outperformed the Cox regression
model in predicting postoperative mortality after hip fracture surgery.
Comparisons of performance indices in the testing dataset also showed
that the ANN model significantly outperformed the Cox model in
sensitivity (0.96 vs.~0.92, p \textless{} 0.001), specificity (0.76
vs.~0.64, p \textless{} 0.001), PPV (0.88 vs.~0.78, p \textless{}
0.001), NPV (0.84 vs.~0.77, p \textless{} 0.001), accuracy (0.93
vs.~0.90, p \textless{} 0.001), and AUROC (0.93 vs.~0.88, p \textless{}
0.001) {[}6{]}. In addition, what makes this study unique is the fact
they are the first to utilize a nationwide population-based dataset for
both training and testing an ANN to forecast outcomes of hip fracture
surgery. As often, in previous studies data often originates from a
single medical center.

While the Cox PH model is widely used in medicine, it is also used in
other fields as well. Previously reliability engineering used
homogeneous Poisson processes and renew process {[}2{]}. However, these
only look at the variable of time-to-the-event, ignoring all other
variables. In the real world, hazard rate of the failure of an item is
influenced by the conditions under which it operates, these conditions,
or covariates, are what the Cox proportional hazard model takes into
account. Proportional hazard models have been used to model component
failure in nuclear plant, ship sonar, aircraft engines, train breaks,
safety valves, aircraft doors, power cables, among other areas {[}2{]}.
This paper also recounts situations that may require covariates that
depend on time: wear on a device, growth of a crack {[}2{]}. If a time
dependent covariate is introduced the model becomes a nonproportional
hazards model and there are additional methods beyond the basic Cox PH
model for handling this.

In summary, the first study stresses the importance of meeting the
proportional hazard assumption in Cox regression analysis, particularly
in the context of ACS patients. The importance of not violating the
assumption and the need for appropriate modeling techniques to address
violations for accurate and reliable prognosis needs to be of the utmost
importance when using Cox regression analysis. The second article sheds
light on regional medical disparities in hypertension complications. The
third article highlights the effectiveness of combining the Cox
regression model with Artificial neural network (ANN). As the Cox
regression model was outperformed by the ANN model when predicting
postoperative mortality after hip fracture surgery. It brings to light
the importance of equitable healthcare access and the introduction of
policies aimed at reducing regional healthcare disparities for better
management of patients with hypertension. Lastly the fourth paper
reviewed here expands the usefulness of the Cox regression model beyond
the medical field to the filed of systems failure and reliability
engineering.



\end{document}
