% Options for packages loaded elsewhere
\PassOptionsToPackage{unicode}{hyperref}
\PassOptionsToPackage{hyphens}{url}
\PassOptionsToPackage{dvipsnames,svgnames,x11names}{xcolor}
%
\documentclass[
  letterpaper,
  DIV=11,
  numbers=noendperiod]{scrreprt}

\usepackage{amsmath,amssymb}
\usepackage{lmodern}
\usepackage{iftex}
\ifPDFTeX
  \usepackage[T1]{fontenc}
  \usepackage[utf8]{inputenc}
  \usepackage{textcomp} % provide euro and other symbols
\else % if luatex or xetex
  \usepackage{unicode-math}
  \defaultfontfeatures{Scale=MatchLowercase}
  \defaultfontfeatures[\rmfamily]{Ligatures=TeX,Scale=1}
\fi
% Use upquote if available, for straight quotes in verbatim environments
\IfFileExists{upquote.sty}{\usepackage{upquote}}{}
\IfFileExists{microtype.sty}{% use microtype if available
  \usepackage[]{microtype}
  \UseMicrotypeSet[protrusion]{basicmath} % disable protrusion for tt fonts
}{}
\makeatletter
\@ifundefined{KOMAClassName}{% if non-KOMA class
  \IfFileExists{parskip.sty}{%
    \usepackage{parskip}
  }{% else
    \setlength{\parindent}{0pt}
    \setlength{\parskip}{6pt plus 2pt minus 1pt}}
}{% if KOMA class
  \KOMAoptions{parskip=half}}
\makeatother
\usepackage{xcolor}
\setlength{\emergencystretch}{3em} % prevent overfull lines
\setcounter{secnumdepth}{-\maxdimen} % remove section numbering
% Make \paragraph and \subparagraph free-standing
\ifx\paragraph\undefined\else
  \let\oldparagraph\paragraph
  \renewcommand{\paragraph}[1]{\oldparagraph{#1}\mbox{}}
\fi
\ifx\subparagraph\undefined\else
  \let\oldsubparagraph\subparagraph
  \renewcommand{\subparagraph}[1]{\oldsubparagraph{#1}\mbox{}}
\fi

\usepackage{color}
\usepackage{fancyvrb}
\newcommand{\VerbBar}{|}
\newcommand{\VERB}{\Verb[commandchars=\\\{\}]}
\DefineVerbatimEnvironment{Highlighting}{Verbatim}{commandchars=\\\{\}}
% Add ',fontsize=\small' for more characters per line
\usepackage{framed}
\definecolor{shadecolor}{RGB}{241,243,245}
\newenvironment{Shaded}{\begin{snugshade}}{\end{snugshade}}
\newcommand{\AlertTok}[1]{\textcolor[rgb]{0.68,0.00,0.00}{#1}}
\newcommand{\AnnotationTok}[1]{\textcolor[rgb]{0.37,0.37,0.37}{#1}}
\newcommand{\AttributeTok}[1]{\textcolor[rgb]{0.40,0.45,0.13}{#1}}
\newcommand{\BaseNTok}[1]{\textcolor[rgb]{0.68,0.00,0.00}{#1}}
\newcommand{\BuiltInTok}[1]{\textcolor[rgb]{0.00,0.23,0.31}{#1}}
\newcommand{\CharTok}[1]{\textcolor[rgb]{0.13,0.47,0.30}{#1}}
\newcommand{\CommentTok}[1]{\textcolor[rgb]{0.37,0.37,0.37}{#1}}
\newcommand{\CommentVarTok}[1]{\textcolor[rgb]{0.37,0.37,0.37}{\textit{#1}}}
\newcommand{\ConstantTok}[1]{\textcolor[rgb]{0.56,0.35,0.01}{#1}}
\newcommand{\ControlFlowTok}[1]{\textcolor[rgb]{0.00,0.23,0.31}{#1}}
\newcommand{\DataTypeTok}[1]{\textcolor[rgb]{0.68,0.00,0.00}{#1}}
\newcommand{\DecValTok}[1]{\textcolor[rgb]{0.68,0.00,0.00}{#1}}
\newcommand{\DocumentationTok}[1]{\textcolor[rgb]{0.37,0.37,0.37}{\textit{#1}}}
\newcommand{\ErrorTok}[1]{\textcolor[rgb]{0.68,0.00,0.00}{#1}}
\newcommand{\ExtensionTok}[1]{\textcolor[rgb]{0.00,0.23,0.31}{#1}}
\newcommand{\FloatTok}[1]{\textcolor[rgb]{0.68,0.00,0.00}{#1}}
\newcommand{\FunctionTok}[1]{\textcolor[rgb]{0.28,0.35,0.67}{#1}}
\newcommand{\ImportTok}[1]{\textcolor[rgb]{0.00,0.46,0.62}{#1}}
\newcommand{\InformationTok}[1]{\textcolor[rgb]{0.37,0.37,0.37}{#1}}
\newcommand{\KeywordTok}[1]{\textcolor[rgb]{0.00,0.23,0.31}{#1}}
\newcommand{\NormalTok}[1]{\textcolor[rgb]{0.00,0.23,0.31}{#1}}
\newcommand{\OperatorTok}[1]{\textcolor[rgb]{0.37,0.37,0.37}{#1}}
\newcommand{\OtherTok}[1]{\textcolor[rgb]{0.00,0.23,0.31}{#1}}
\newcommand{\PreprocessorTok}[1]{\textcolor[rgb]{0.68,0.00,0.00}{#1}}
\newcommand{\RegionMarkerTok}[1]{\textcolor[rgb]{0.00,0.23,0.31}{#1}}
\newcommand{\SpecialCharTok}[1]{\textcolor[rgb]{0.37,0.37,0.37}{#1}}
\newcommand{\SpecialStringTok}[1]{\textcolor[rgb]{0.13,0.47,0.30}{#1}}
\newcommand{\StringTok}[1]{\textcolor[rgb]{0.13,0.47,0.30}{#1}}
\newcommand{\VariableTok}[1]{\textcolor[rgb]{0.07,0.07,0.07}{#1}}
\newcommand{\VerbatimStringTok}[1]{\textcolor[rgb]{0.13,0.47,0.30}{#1}}
\newcommand{\WarningTok}[1]{\textcolor[rgb]{0.37,0.37,0.37}{\textit{#1}}}

\providecommand{\tightlist}{%
  \setlength{\itemsep}{0pt}\setlength{\parskip}{0pt}}\usepackage{longtable,booktabs,array}
\usepackage{calc} % for calculating minipage widths
% Correct order of tables after \paragraph or \subparagraph
\usepackage{etoolbox}
\makeatletter
\patchcmd\longtable{\par}{\if@noskipsec\mbox{}\fi\par}{}{}
\makeatother
% Allow footnotes in longtable head/foot
\IfFileExists{footnotehyper.sty}{\usepackage{footnotehyper}}{\usepackage{footnote}}
\makesavenoteenv{longtable}
\usepackage{graphicx}
\makeatletter
\def\maxwidth{\ifdim\Gin@nat@width>\linewidth\linewidth\else\Gin@nat@width\fi}
\def\maxheight{\ifdim\Gin@nat@height>\textheight\textheight\else\Gin@nat@height\fi}
\makeatother
% Scale images if necessary, so that they will not overflow the page
% margins by default, and it is still possible to overwrite the defaults
% using explicit options in \includegraphics[width, height, ...]{}
\setkeys{Gin}{width=\maxwidth,height=\maxheight,keepaspectratio}
% Set default figure placement to htbp
\makeatletter
\def\fps@figure{htbp}
\makeatother

\KOMAoption{captions}{tableheading}
\makeatletter
\makeatother
\makeatletter
\makeatother
\makeatletter
\@ifpackageloaded{caption}{}{\usepackage{caption}}
\AtBeginDocument{%
\ifdefined\contentsname
  \renewcommand*\contentsname{Table of contents}
\else
  \newcommand\contentsname{Table of contents}
\fi
\ifdefined\listfigurename
  \renewcommand*\listfigurename{List of Figures}
\else
  \newcommand\listfigurename{List of Figures}
\fi
\ifdefined\listtablename
  \renewcommand*\listtablename{List of Tables}
\else
  \newcommand\listtablename{List of Tables}
\fi
\ifdefined\figurename
  \renewcommand*\figurename{Figure}
\else
  \newcommand\figurename{Figure}
\fi
\ifdefined\tablename
  \renewcommand*\tablename{Table}
\else
  \newcommand\tablename{Table}
\fi
}
\@ifpackageloaded{float}{}{\usepackage{float}}
\floatstyle{ruled}
\@ifundefined{c@chapter}{\newfloat{codelisting}{h}{lop}}{\newfloat{codelisting}{h}{lop}[chapter]}
\floatname{codelisting}{Listing}
\newcommand*\listoflistings{\listof{codelisting}{List of Listings}}
\makeatother
\makeatletter
\@ifpackageloaded{caption}{}{\usepackage{caption}}
\@ifpackageloaded{subcaption}{}{\usepackage{subcaption}}
\makeatother
\makeatletter
\@ifpackageloaded{tcolorbox}{}{\usepackage[many]{tcolorbox}}
\makeatother
\makeatletter
\@ifundefined{shadecolor}{\definecolor{shadecolor}{rgb}{.97, .97, .97}}
\makeatother
\makeatletter
\makeatother
\ifLuaTeX
  \usepackage{selnolig}  % disable illegal ligatures
\fi
\IfFileExists{bookmark.sty}{\usepackage{bookmark}}{\usepackage{hyperref}}
\IfFileExists{xurl.sty}{\usepackage{xurl}}{} % add URL line breaks if available
\urlstyle{same} % disable monospaced font for URLs
\hypersetup{
  colorlinks=true,
  linkcolor={blue},
  filecolor={Maroon},
  citecolor={Blue},
  urlcolor={Blue},
  pdfcreator={LaTeX via pandoc}}

\author{}
\date{}

\begin{document}
\ifdefined\Shaded\renewenvironment{Shaded}{\begin{tcolorbox}[interior hidden, boxrule=0pt, frame hidden, breakable, enhanced, borderline west={3pt}{0pt}{shadecolor}, sharp corners]}{\end{tcolorbox}}\fi

\begin{Shaded}
\begin{Highlighting}[]
\FunctionTok{library}\NormalTok{(survival)}
\CommentTok{\#install.packages("survminer")}
\FunctionTok{library}\NormalTok{(survminer)}
\end{Highlighting}
\end{Shaded}

\begin{verbatim}
Loading required package: ggplot2
\end{verbatim}

\begin{verbatim}
Loading required package: ggpubr
\end{verbatim}

\begin{verbatim}

Attaching package: 'survminer'
\end{verbatim}

\begin{verbatim}
The following object is masked from 'package:survival':

    myeloma
\end{verbatim}

\begin{Shaded}
\begin{Highlighting}[]
\NormalTok{heart\_transplant }\OtherTok{\textless{}{-}} \FunctionTok{read.csv}\NormalTok{(}\StringTok{"./heart\_transplant.csv"}\NormalTok{)}
\CommentTok{\#generate numeric columns for survived, prior, and transplant from existing charachter columns}
\NormalTok{heart\_transplant}\SpecialCharTok{$}\NormalTok{survived2 }\OtherTok{\textless{}{-}} \FunctionTok{as.numeric}\NormalTok{(}\FunctionTok{ifelse}\NormalTok{(heart\_transplant}\SpecialCharTok{$}\NormalTok{survived}\SpecialCharTok{==}\StringTok{"dead"}\NormalTok{,}\DecValTok{1}\NormalTok{,}\DecValTok{0}\NormalTok{)) }\SpecialCharTok{\textgreater{}} \DecValTok{0}
\NormalTok{heart\_transplant}\SpecialCharTok{$}\NormalTok{prior2 }\OtherTok{\textless{}{-}} \FunctionTok{as.numeric}\NormalTok{(}\FunctionTok{ifelse}\NormalTok{(heart\_transplant}\SpecialCharTok{$}\NormalTok{prior}\SpecialCharTok{==}\StringTok{"yes"}\NormalTok{,}\DecValTok{1}\NormalTok{,}\DecValTok{0}\NormalTok{))}
\NormalTok{heart\_transplant}\SpecialCharTok{$}\NormalTok{transplant2 }\OtherTok{\textless{}{-}} \FunctionTok{as.numeric}\NormalTok{(}\FunctionTok{ifelse}\NormalTok{(heart\_transplant}\SpecialCharTok{$}\NormalTok{transplant}\SpecialCharTok{==}\StringTok{"treatment"}\NormalTok{,}\DecValTok{1}\NormalTok{,}\DecValTok{0}\NormalTok{))}
\NormalTok{all\_strTrtmnt.mod }\OtherTok{\textless{}{-}} \FunctionTok{coxph}\NormalTok{(}\FunctionTok{Surv}\NormalTok{(survtime, survived2)}\SpecialCharTok{\textasciitilde{}}\NormalTok{ age }\SpecialCharTok{+}\NormalTok{ prior2 }\SpecialCharTok{+} \FunctionTok{strata}\NormalTok{(transplant2), }\AttributeTok{data=}\NormalTok{heart\_transplant)}
\FunctionTok{summary}\NormalTok{(all\_strTrtmnt.mod)}
\end{Highlighting}
\end{Shaded}

\begin{verbatim}
Call:
coxph(formula = Surv(survtime, survived2) ~ age + prior2 + strata(transplant2), 
    data = heart_transplant)

  n= 99, number of events= 71 

           coef exp(coef) se(coef)      z Pr(>|z|)   
age     0.04662   1.04773  0.01434  3.250  0.00115 **
prior2 -0.78988   0.45390  0.44480 -1.776  0.07576 . 
---
Signif. codes:  0 '***' 0.001 '**' 0.01 '*' 0.05 '.' 0.1 ' ' 1

       exp(coef) exp(-coef) lower .95 upper .95
age       1.0477     0.9544    1.0187     1.078
prior2    0.4539     2.2031    0.1898     1.085

Concordance= 0.642  (se = 0.038 )
Likelihood ratio test= 15.95  on 2 df,   p=3e-04
Wald test            = 13.96  on 2 df,   p=9e-04
Score (logrank) test = 14.36  on 2 df,   p=8e-04
\end{verbatim}

\hypertarget{results}{%
\chapter{Results}\label{results}}

Let's look at the data, Image 1 best illustrates this answer. The rate
of death with no heart transplant drops significantly over time. The
number of patients who did not receive a transplant is 32, with only 2
surviving past 500 days. Patients who did receive the transplant are 67,
with more than 22 surviving past 500 days. In Image 2 categories
patients under 40, between 40-50 and over 50 years old and a hazard
ratio of 2.08. From one category to the next this tells us you are twice
as likely to die. Not accounting for if you get the transplant or not.
First model excluded wait time as wait time is only available for
patients who received transplant, and not all patients in the study had
the opportunity to receive a heart transplant. The C-statistic =0.747
(see Table 1). \#\# Code \#generate model without wait time \#wait time
is only available for those who received transplant

\begin{Shaded}
\begin{Highlighting}[]
\NormalTok{all.mod }\OtherTok{\textless{}{-}} \FunctionTok{coxph}\NormalTok{(}\FunctionTok{Surv}\NormalTok{(survtime, survived2)}\SpecialCharTok{\textasciitilde{}}\NormalTok{ acceptyear }\SpecialCharTok{+}\NormalTok{ age }\SpecialCharTok{+}\NormalTok{ prior }\SpecialCharTok{+}\NormalTok{ transplant, }\AttributeTok{data=}\NormalTok{heart\_transplant)}
\end{Highlighting}
\end{Shaded}

Second model generated included wait time to see if wait time is
significant, only looking at records that received the transplant, not
the control who did not receive transplant. The result C-statistic
=0.683 (see Table 2).

\hypertarget{code}{%
\section{Code}\label{code}}

\begin{Shaded}
\begin{Highlighting}[]
\NormalTok{treated.mod }\OtherTok{\textless{}{-}} \FunctionTok{coxph}\NormalTok{(}\FunctionTok{Surv}\NormalTok{(survtime, survived2)}\SpecialCharTok{\textasciitilde{}}\NormalTok{ acceptyear }\SpecialCharTok{+}\NormalTok{ age }\SpecialCharTok{+}\NormalTok{ prior }\SpecialCharTok{+}\NormalTok{ wait, }\AttributeTok{data=}\NormalTok{heart\_transplant)}
\end{Highlighting}
\end{Shaded}

Third model without an accepted year was generated and then compared
using an ANOVA table (see Table 4). The p-value = 0.1656 which is
greater than 0.05 and we can drop the accepted year as a variable.

\hypertarget{code-1}{%
\section{Code}\label{code-1}}

\begin{Shaded}
\begin{Highlighting}[]
\CommentTok{\#generate model without accepted year}
\NormalTok{all2.mod }\OtherTok{\textless{}{-}} \FunctionTok{coxph}\NormalTok{(}\FunctionTok{Surv}\NormalTok{(survtime, survived2)}\SpecialCharTok{\textasciitilde{}}\NormalTok{ age }\SpecialCharTok{+}\NormalTok{ prior }\SpecialCharTok{+}\NormalTok{ transplant, }\AttributeTok{data=}\NormalTok{heart\_transplant)}
\end{Highlighting}
\end{Shaded}

The C-statistic for the model without an accepted year is equal to 0.739
(see Table 3). Which shows that this model has a higher predictability
capabilities compared to the model including the accepted year with a
C-statistic = 0.683.

\hypertarget{code-2}{%
\section{Code}\label{code-2}}

\begin{Shaded}
\begin{Highlighting}[]
\CommentTok{\#examine model}
\FunctionTok{summary}\NormalTok{(all2.mod)}
\end{Highlighting}
\end{Shaded}

\begin{verbatim}
Call:
coxph(formula = Surv(survtime, survived2) ~ age + prior + transplant, 
    data = heart_transplant)

  n= 99, number of events= 71 

                        coef exp(coef) se(coef)      z Pr(>|z|)    
age                  0.05647   1.05810  0.01457  3.875 0.000107 ***
prioryes            -0.70778   0.49274  0.44368 -1.595 0.110655    
transplanttreatment -1.72587   0.17802  0.28398 -6.077 1.22e-09 ***
---
Signif. codes:  0 '***' 0.001 '**' 0.01 '*' 0.05 '.' 0.1 ' ' 1

                    exp(coef) exp(-coef) lower .95 upper .95
age                    1.0581     0.9451    1.0283    1.0888
prioryes               0.4927     2.0295    0.2065    1.1756
transplanttreatment    0.1780     5.6174    0.1020    0.3106

Concordance= 0.746  (se = 0.031 )
Likelihood ratio test= 47.31  on 3 df,   p=3e-10
Wald test            = 48.37  on 3 df,   p=2e-10
Score (logrank) test = 55.28  on 3 df,   p=6e-12
\end{verbatim}

\begin{Shaded}
\begin{Highlighting}[]
\CommentTok{\#compare model that includes acceptedyear and that does not}
\CommentTok{\#high p value, there is not a statistically significant difference between the two models, can}
\CommentTok{\#drop the variable}
\FunctionTok{anova}\NormalTok{(all.mod, all2.mod, }\AttributeTok{test=}\StringTok{"LRT"}\NormalTok{)}
\end{Highlighting}
\end{Shaded}

\begin{verbatim}
Analysis of Deviance Table
 Cox model: response is  Surv(survtime, survived2)
 Model 1: ~ acceptyear + age + prior + transplant
 Model 2: ~ age + prior + transplant
   loglik  Chisq Df Pr(>|Chi|)
1 -255.49                     
2 -256.28 1.5784  1      0.209
\end{verbatim}

The C-statistic for the model without an accepted year is equal to 0.739
(see Table 3). Which shows that this model has a higher predictability
capabilities compared to the model including the accepted year with a
C-statistic = 0.683.

\hypertarget{discussion}{%
\chapter{Discussion}\label{discussion}}

The key question at the heart of our study revolves around a fundamental
question healthcare: does a heart transplant prolong the life of
patients? Logically, one would assume that undergoing such a significant
medical procedure, with all the risk involved and complexities, should
lead to an extended life of the individual. After all, why would
patients willingly choose to undergo heart transplants if they did not
anticipate the promise of improved survival and a better quality of
life? To address this key question, we delved into the data provided by
Stanford Heart Transplant Study, utilizing the Cox Regression model to
shed light on the matter. Our analysis, as portrayed in Image 1,
provides a compelling answer to the question. In this graph, patients
who received a heart transplant are represented in teal, while those who
did not receive a transplant are represented in red. The important
observation is the remarkable decrease in the rate of death among those
who did not undergo a heart transplant over time. The total sample size
was 99 patients, 32 of them did not receive a transplant while 67 others
did receive the heart transplant. From the 32 who did not receive a
transplant only 2 patients managed to survive past 500 days. In stark
contrast, among the 67 patients who did receive the transplant, more
than 22 of them survived beyond the 500-day mark. This visually striking
contrast suggests that heart transplantation does indeed have a
significant impact on extending the lives of patients in this context.
Image 2, on the other hand, offers a different perspective by
categorizing patients based on age groups; those under 40, those between
40 and 50, and those over 50 years old. The calculated hazard ratio of
2.08 when moving from one age category to the next is a notable finding.
This ratio indicates that, all else being equal, patients in a higher
age category are approximately twice as likely to face mortality risks.
This observation is irrespective of whether they receive a heart
transplant or not. In a border sense, this implies that, within the
confines of the criteria for inclusion in the study, the need for a
heart transplant--advancing from one age group to the next essentially
doubles the risk of mortality. This is entirely consistent with our
understanding of the aging process and its impact on an individual's
overall health. As one grows older, the body's ability to withstand the
rigors of a medical condition that necessitates a heart transplant
diminishes. The data thus reinforces the notion that age itself is a
critical factor in determining patients outcomes. In our group's pursuit
of identifying the most robust and predictive model, we turned to the
concordance statistics, specifically the C-statistic, as a guiding
metric for a model selection. The C-statistic serves as a valuable tool
for assessing the discriminatory power of predictive models, aiding in
our ability to distinguish their aptitude for accurately ranking risks
and predicting outcomes. Our findings are as follows:
\(\textbf{Model 1: Excluding Wait Time}\) Our initial model was
constructed without considering wait time as a variable. This decision
was rooted in the understanding that wait time data was only available
for patients who ultimately received a heart transplant, excluding those
who did not have the opportunity to undergo the procedure. The
C-statistic for this model, as presented in Table 1, was calculated to
be 0.747. This initial C-statistic provided an encouraging start for our
analysis.

\textbf{Model 2: Including Wait Time for Transplant Recipients}

The second model purpose was to assess the significance of wait time as
a predictor of outcomes but was excluded as an actual model due to the
fact that it only applies to a subgroup of patients who received a heart
transplant. In this instance, the model excluded the control group. The
C-statistic for this model, focusing solely on transplant recipients,
was 0.683, as detailed in Table 2. However, it's important to note that
this model's utility is limited due to its omission of the control
group, which restricts its generalizability and practicality. Model 3:
Variable Selection and ANOVA Table Continuing our goal for the most
informative model, we developed a third model. In this iteration, we
sought to explore the influence of the ``accepted year'' variable on
predictive capabilities. To make an informed decision regarding the
necessity of including this variable, we meticulously assessed its
impact using an ANOVA table (as shown in Table 4). The analysis provided
us with a p-value of 0.1656, exceeding the common significance level of
0.05. As a result, we concluded that ``accepted year'' could be excluded
as a variable in the model.

\textbf{Model 3: Without Accepted Year} With ``accepted year'' removed
as a variable, we obtained a C-statistic of 0.739 for this refined
model, as indicated in Table. The improved C-statistic underscores the
enhanced predictive capabilities of this modelwhen compare to the
previous iteration, which included ``accepted year'' and had a
C-statistic of 0.683

\hypertarget{conclusion}{%
\chapter{Conclusion}\label{conclusion}}

In conclusion, our research provides strong support that heart
transplantation has a positive and significant impact on extending the
lives of patients. The data was provided by Stanford Heart Transplant
Study. With the help of a survival analysis tool: Cox Regression model.
The Cox regression was best suited for this study due to the fact that
it is suitable for time-to-event outcomes. In this case, how much longer
does a patient live if given a heart transplant versus those who do not
receive the transplant. In some cases patients post transplant lived an
extra 1-4 years. The best model used to support the data and build the
images was Model 3. Future research could look into which age group had
best survivability post transplant. This could provide healthcare
workers valuable information regarding who should get the heart
transplant if there are patients in different age groups who are waiting
for a new heart and which would benefit the most and survive. As we know
donated organs are not easy to come by and making sure that an organ
such as a heart should go to the best candidate that would live longer
with that new heart. Our findings contribute to a deeper understanding
of the complex interplay between heart transplantation, age and patient
outcomes in the realm of cardiac healthcare.

\textbf{Table 1: With variables: Accepted Year, Age, Prior, and Transplant}

\begin{Shaded}
\begin{Highlighting}[]
\FunctionTok{summary}\NormalTok{(all.mod)}
\end{Highlighting}
\end{Shaded}

\begin{verbatim}
Call:
coxph(formula = Surv(survtime, survived2) ~ acceptyear + age + 
    prior + transplant, data = heart_transplant)

  n= 99, number of events= 71 

                        coef exp(coef) se(coef)      z Pr(>|z|)    
acceptyear          -0.08294   0.92040  0.06647 -1.248 0.212083    
age                  0.05553   1.05711  0.01432  3.878 0.000105 ***
prioryes            -0.63345   0.53076  0.44818 -1.413 0.157542    
transplanttreatment -1.69836   0.18298  0.28791 -5.899 3.66e-09 ***
---
Signif. codes:  0 '***' 0.001 '**' 0.01 '*' 0.05 '.' 0.1 ' ' 1

                    exp(coef) exp(-coef) lower .95 upper .95
acceptyear             0.9204      1.086    0.8080    1.0485
age                    1.0571      0.946    1.0278    1.0872
prioryes               0.5308      1.884    0.2205    1.2776
transplanttreatment    0.1830      5.465    0.1041    0.3217

Concordance= 0.756  (se = 0.031 )
Likelihood ratio test= 48.89  on 4 df,   p=6e-10
Wald test            = 49.36  on 4 df,   p=5e-10
Score (logrank) test = 55.91  on 4 df,   p=2e-11
\end{verbatim}

\textbf{Table 2: With Variables Accepted Year, Age, Prior, and Wait Time}

\begin{Shaded}
\begin{Highlighting}[]
\FunctionTok{summary}\NormalTok{(treated.mod)}
\end{Highlighting}
\end{Shaded}

\begin{verbatim}
Call:
coxph(formula = Surv(survtime, survived2) ~ acceptyear + age + 
    prior + wait, data = heart_transplant)

  n= 67, number of events= 43 
   (32 observations deleted due to missingness)

                coef exp(coef)  se(coef)      z Pr(>|z|)  
acceptyear  0.005578  1.005594  0.095776  0.058   0.9536  
age         0.050544  1.051843  0.021308  2.372   0.0177 *
prioryes   -0.751733  0.471549  0.453454 -1.658   0.0974 .
wait       -0.008337  0.991697  0.005109 -1.632   0.1027  
---
Signif. codes:  0 '***' 0.001 '**' 0.01 '*' 0.05 '.' 0.1 ' ' 1

           exp(coef) exp(-coef) lower .95 upper .95
acceptyear    1.0056     0.9944    0.8335     1.213
age           1.0518     0.9507    1.0088     1.097
prioryes      0.4715     2.1207    0.1939     1.147
wait          0.9917     1.0084    0.9818     1.002

Concordance= 0.685  (se = 0.043 )
Likelihood ratio test= 14.23  on 4 df,   p=0.007
Wald test            = 12.64  on 4 df,   p=0.01
Score (logrank) test = 13.19  on 4 df,   p=0.01
\end{verbatim}

\textbf{Table 3: Model with Age, Prior, Transplant and Without Accepted Year}

\begin{Shaded}
\begin{Highlighting}[]
\FunctionTok{summary}\NormalTok{(all2.mod)}
\end{Highlighting}
\end{Shaded}

\begin{verbatim}
Call:
coxph(formula = Surv(survtime, survived2) ~ age + prior + transplant, 
    data = heart_transplant)

  n= 99, number of events= 71 

                        coef exp(coef) se(coef)      z Pr(>|z|)    
age                  0.05647   1.05810  0.01457  3.875 0.000107 ***
prioryes            -0.70778   0.49274  0.44368 -1.595 0.110655    
transplanttreatment -1.72587   0.17802  0.28398 -6.077 1.22e-09 ***
---
Signif. codes:  0 '***' 0.001 '**' 0.01 '*' 0.05 '.' 0.1 ' ' 1

                    exp(coef) exp(-coef) lower .95 upper .95
age                    1.0581     0.9451    1.0283    1.0888
prioryes               0.4927     2.0295    0.2065    1.1756
transplanttreatment    0.1780     5.6174    0.1020    0.3106

Concordance= 0.746  (se = 0.031 )
Likelihood ratio test= 47.31  on 3 df,   p=3e-10
Wald test            = 48.37  on 3 df,   p=2e-10
Score (logrank) test = 55.28  on 3 df,   p=6e-12
\end{verbatim}

\textbf{Table 4: ANOVA Table Comparing Models 1 & 2}

\begin{Shaded}
\begin{Highlighting}[]
\FunctionTok{anova}\NormalTok{(all.mod, all2.mod, }\AttributeTok{test=}\StringTok{"LRT"}\NormalTok{)}
\end{Highlighting}
\end{Shaded}

\begin{verbatim}
Analysis of Deviance Table
 Cox model: response is  Surv(survtime, survived2)
 Model 1: ~ acceptyear + age + prior + transplant
 Model 2: ~ age + prior + transplant
   loglik  Chisq Df Pr(>|Chi|)
1 -255.49                     
2 -256.28 1.5784  1      0.209
\end{verbatim}



\end{document}
